%        File: lickly_hw1.tex
%      Author: Ben Lickly
%     Created: Tue Sep 29 04:00 PM 2009 P
%
\documentclass{amsart}
\title{EECS 219C: HW \#1}
\author{Ben Lickly}
\date{September 30, 2009}

\usepackage{algorithm2e}

\newcommand{\no}[1]{\overline{#1}}
\newcommand{\lef}[1]{\overrightarrow{#1}}
\newcommand{\ri}[1]{\overleftarrow{#1}}
\newcommand{\po}[1]{\overbrace{#1}}

\begin{document}
\begin{enumerate}
  \item Renamable Horn-SAT
    \begin{enumerate}
      \item Give a linear-time algorithm to decide the satisfiability of
	HornSAT formulas.

Let the number of variables be $n$.
\begin{algorithm}
$solution \leftarrow \operatorname{False}^n$ \;
\While{there exists a positive unit clause $x_j$} {
  \If {there exists a negative unit clause $\neg x_j$} {
     return ``Unsatisfiable''
  }
  $solution \leftarrow solution[x_j = \operatorname{True}]$ \;
  Remove all clauses that contain $x_j$ \;
  Remove $\neg x_j$ from all remaining clauses \;
}
return  $solution$
\end{algorithm}
Since this algorithm only considers each literal at most once,
its runtime is linear in the total number of literals.
      \item Give a polynomial-time algorithm to check whether a formula on 
	$n$ variables comprising $m$ CNF clauses is renamable Horn.

First, take each of the literals in the $i$th clause named as follows:
\[
C_i = L_{i,1} \vee L_{i,2} \vee \cdots \vee L_{i,len(C_i)}
\]
In a Horn clause, at most one literal in each clause can be unnegated.
Now, simply note that in order to be renamable Horn, there must be some
literals that can be swapped with their complements in order to make
the problem HornSAT.

The trick is we can form a 2-SAT problem\footnote{Idea to form a 2-SAT problem inspired by~\cite{322059}} where each variable represents
a variable that could be negated:
\[
\bigwedge_{i=1}^m \bigwedge_{j \ne k \in C_i} L_{i,j} \vee L_{i,k}
\]
Here, if we find a satisfying assignment, then the original problem
was renamable HornSAT, where every variable that was assigned True
needs to be swapped. Clearly, this 2-SAT problem is polynomial in the size
of the original problem (easily bounded by $O(m \times n^2)$), and since 2-SAT
is linear time solvable, our entire algorithm is polynomial time.
    \end{enumerate}
  \item DLL Algorithm and Exponential Search
Claim: No matter which variable we branch on, the total number of branches
will be exponential in $n$, the number of pigeons.
\begin{proof}
Let $x_{ij}$ be an arbitrarily chosen variable corresponding to putting bird
$i$ in hole $j$.
We will claim that no matter which way this variable is branched, the
resulting problem will be at least as big as an $n-1$ sized version.

Case 1: $x_{ij} = true$ \\
In this case, we can discard from the first set of constraints
the constraint that bird $i$ needs a hole.
We can infer from the second set of constraints that no other bird
can be fit in hole $j$.
Thus, we are left with exactly the problem of fitting $n-1$ pigeons
in $n-2$ holes, which is exactly the $n-1$ sized problem.

Case 2: $x_{ij} = false$ \\
In this case, we assume that pigeon $i$ does not fit in hole $j$.
By the constraint in the first set that refers to pigeon $i$, we are left
with $n-2$ potential holes for pigeon $i$.
Certainly, it would not make the problem more difficult if we had an oracle
that alerted us that certain branchings lead to an unsatisfiable (since
this only serves to prune our search space).
In particular, consider what would happen if we had an oracle that
told us that placing the pigeon in $n-3$ of the remaining holes would result
in an unsatisfiable problem.
This would lead to a unit clause that forced the pigeon to be in the final
hole, leaving us with exactly the $n-1$ sized problem.

Since the overall problem is unsatisfiable, both branches on
$x_{ij}$ need to be explored.
This leads to the recurrence $T(n) = 2\times T(n-1)$,
whose solution is $O(2^n)$.
\end{proof}
  \item Experimenting with a SAT Solver
  \item Comparing head-tail pointers with 2-literal watching
    \begin{enumerate}
      \item For the Head-Tail scheme, state how one detects if the clause becomes a unit clause and a conflict clause.

        One detects a unit clause when the head literal and tail literal are
        the same. A conflict clause is created when this literal is negated.
      \item Consider the following clause:
        \begin{align*}
          (x_1 \, \no{x_2} \, x_4 \, \no{x_6} \, x_7 \, x_{10} \, x_{15})
        \end{align*}
SATO's Head-Tail scheme:
\begin{align*}
1.\qquad&(x_1\, \lef{\no{x_2}}\, x_4\, \no{x_6}\, x_7\, x_{10}\, \ri{x_{15}})\\
2.\qquad&(x_1\, \lef{\no{x_2}}\, x_4\, \no{x_6}\, x_7\, x_{10}\, \ri{x_{15}})\\
3.\qquad&(x_1\, \lef{\no{x_2}}\, x_4\, \ri{\no{x_6}}\, x_7\, x_{10}\, x_{15})\\
4.\qquad&(x_1\, \no{x_2}\, x_4\, \lef{\ri{\no{x_6}}}\, x_7\, x_{10}\, x_{15})
\text{ (Unit clause implies $x_6 = 0$)} \\
5.\qquad&(x_1\, \lef{\no{x_2}}\, x_4\, \no{x_6}\, x_7\, x_{10}\, \ri{x_{15}})\\
6.\qquad&(x_1\, \lef{\no{x_2}}\, x_4\, \no{x_6}\, x_7\, x_{10}\, \ri{x_{15}})
\text{ (Clause is now satisfied by $x_2$ assignment)} \\
\end{align*}
Chaff’s 2-literal watching:
\begin{align*}
1.\qquad&(x_1\, \po{\no{x_2}}\, \po{x_4}\, \no{x_6}\, x_7\, x_{10}\, x_{15})\\
2.\qquad&(x_1\, \po{\no{x_2}}\, x_4\, \po{\no{x_6}}\, x_7\, x_{10}\, x_{15})\\
3.\qquad&(x_1\, \po{\no{x_2}}\, x_4\, \po{\no{x_6}}\, x_7\, x_{10}\, x_{15})\\
4.\qquad&(x_1\, \no{x_2}\, x_4\, \po{\no{x_6}}\, x_7\, x_{10}\, x_{15})
\text{ (Unit clause implies $x_6 = 0$)} \\
5.\qquad&(x_1\, \po{\no{x_2}}\, \po{x_4}\, \no{x_6}\, x_7\, x_{10}\, x_{15})\\
5.\qquad&(x_1\, \po{\no{x_2}}\, \po{x_4}\, \no{x_6}\, x_7\, x_{10}\, x_{15})
\text{ (Clause is now satisfied by $x_2$ assignment)} \\
\end{align*}
    \end{enumerate}
  \item BDDs

    Claim: The number of nodes in the $i$th layer of the BDD of a symmetric
    boolean function is $i$.  Thus, the total number of nodes in the BDD is
    $\frac{n (n+1)}{2}$, which is $O(n^2)$.
\begin{proof}
  (By induction)

  Base case: The binary decision diagram for a 1-argument function is has size
  1, for the single variable.

  Inductive case:  Assume that there are $i$ nodes in the $i$th layer of a BDD.
  Clearly, there are $2i$ edges that come out of this layer.
  Since the function is symmetric, however, not all of these edges have unique
  states.  In particular, only two edges (corresponding to all 0s and all 1s)
  will be unique.  The remaining $2i - 2$ edges will have two edges going into
  each node, meaning that the $i+1$st layer will have $\frac{2i -2}{2} + 2$
  nodes, which is exactly $i+1$.
\end{proof}
\end{enumerate}

\bibliographystyle{plain}
\bibliography{lickly_hw1}
\end{document}

