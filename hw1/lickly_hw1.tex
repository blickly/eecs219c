%        File: lickly_hw1.tex
%      Author: Ben Lickly
%     Created: Tue Sep 29 04:00 PM 2009 P
%
\documentclass{amsart}
\title{EECS 219C: HW \#1}
\author{Ben Lickly}
\date{September 30, 2009}
\begin{document}

\begin{enumerate}
  \item Renamable Horn-SAT
  \item DLL Algorithm and Exponential Search
  \item Experimenting with a SAT Solver
  \item Comparing head-tail pointers with 2-literal watching
    \begin{enumerate}
      \item Show how one tracks literals
      \item Perform \ldots
    \end{enumerate}
  \item BDDs
    Claim: The number of nodes in the $i$th layer of the BDD of a symmetric
    boolean function is $i$.  Thus, the total number of nodes in the BDD is
    $\frac{n (n+1)}{2}$, which is $O(n^2)$.
\begin{proof}
  (By induction)

  Base case: The binary decision diagram for a 1-argument function is has size
  1, for the single variable.

  Inductive case:  Assume that there are $i$ nodes in the $i$th layer of a BDD.
  Clearly, there are $2i$ edges that come out of this layer.
  Since the function is symmetric, however, not all of these edges have unique
  states.  In particular, only two edges (corresponding to all 0s and all 1s)
  will be unique.  The remaining $2i - 2$ edges will have two edges going into
  each node, meaning that the $i+1$st layer will have $\frac{2i -2}{2} + 2$
  nodes, which is exactly $i+1$.
\end{proof}
\end{enumerate}

\end{document}

